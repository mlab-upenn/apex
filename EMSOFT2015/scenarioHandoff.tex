\subsection{Handoff between scenarios}
\label{scenarioHandoff}

The agents in a scenario need to be initialized when a scenario starts, and $Init$ gives the set of valid initializations for each agent's state.
Consider a mission $M = s_1 s_2$ with $s_i = (\agentInstanceSet_i,\behavior_i, \Lc_i, Init_i, \exitConditions_i)$, $i=1,2$,
and where the state of $s_1$ maps to that of $s_2$ by a map $\pi$.
the exit conditions $\exitConditions_0$ initializes $s_1$. 
I.e. if $x_k$ is the state of the $k^{th}$ agent in $A_2$, then the initial set for $x_k$ is $\pi(L(\exitConditions_1))$,
where $L(\exitConditions_1)$ is the union of the languages of the formulae in $\exitConditions_1$, i.e. 
\[L(\exitConditions_1) = \cup_{\formula \in \exitConditions_1}L(\formula)\]
For this to be a valid initialization, it must hold that 
\begin{equation}
\label{eq:exitInitialize}
\pi(L(\exitConditions_1)) \subset Init
\end{equation}

Another condition for a valid handoff is that no new laws enter in effect in the new scenario which immediately invalidate the current behavior of the vehicle, i.e. it must hold that 
\begin{equation}
\label{eq:lawInitialize}
x_1(t_{handoff}) \models \lawSet_1 \implies x_1(t_{handoff} \models \lawSet_2)
\end{equation}
where $t_{handoff}$ is the time at which scenario $s_1$ hands off to scenario $s_2$.

Note that \eqref{eq:exitInitialize}, \eqref{eq:lawInitialize} are not verification assumptions.
Rather, they are part of what needs to be verified. 
A violation of either one indicates that the mission can not be completed as described.

\begin{exmp}[Lane change contiued]
	
	\end{exmp}

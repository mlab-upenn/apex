\subsection{The benefits of scenario decomposition}
\label{mixingFormalisms}

By decomposing each mission into scenarios, and defining valid handoff conditions between scenarios, we are free to choose the best formalism and tools available for verifying each scenario.
For example, consider the Lane Change running example, where there is a transition between a DriveStraight scenario and a ChangeLane scenario.
The concrete car dynamics for the ego vehicle are given by a forced 7-dimensional nonlinear ODE, 
Direct formal verification of this model is not possible. 
However, during the DriveStraight scenario, it is possible to over-approximate these dynamics, and run a timed automata verification tool to model check the resulting system.
This is done in Section \ref{caseStudy}.

In the ChangeLane scenario, bowever, the accuracy of the concrete, un-approximated dynamics is necessary.
The over-approximation done in DriveStraight scenario becomes too conservative. 
In such a scenario, we can run a reachability tool to verify the correctness of the vehicle's behavior.
Even though reachability analysis is, in general, an expensive operation, we are only running it to verify select scenarios that require the complex concrete dynamics.
Thus, by decomposing a mission into scenarios, then correctly composing the scenarios' verification results, we can verify the entire mission.
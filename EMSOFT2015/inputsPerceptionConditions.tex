\subsubsection{Inputs and perception conditions}
\label{inputsPerceptionConditions}
Without the perception conditions $C$, and without the inputs, a CHA is simply a hybrid automaton~\cite{Henzinger96}.
The inputs are used to model two phenomena: 
first, some inputs will model the output of the perception layer of the system. 
\todo[inline]{See fig. -- tartan figure from intro -- }
The perception layer of agent $A$ interprets the perceptible aspects of other agents, i.e. $\funcOut_{A'}(\stPt)$ for $A'\neq A$.
Depending on the abstraction level at which they are considered, these inputs may be discrete (e.g., a boolean to indicate the presence of a car in the current lane) or continuous (e.g., the estimated position of a pedestrian).
Second, other inputs will model control commands to the vehicle issued by its own autonomous controllers. 
E.g. a controller may cause the vehicle to switch modes: such inputs are discrete. 
Another controller (at a different level of abstraction) produces a piecewise continuous throttle angle command. Such an input is continuous. 

To model how agents perceive each other at a high level, we consider that certain aspects of an agent can occasionally be perceived by other agents.
The simplest such aspect is the agent's visual appearance.
Formally, associated to an agent is a function $\funcOut:\reals^n \rightarrow \reals^p$.
At each time instant $(t,j)$, the agent broadcasts $\funcOut(\stPt(t,j))$.
Not every other agent can perceive this broadcast.
Given two agents $A_1$ and $A_2$ in the same scenario, with agent $A_1$ broadcasting $y = f(x)$, agent $A_2$ must meet certain conditions on its state to be able to receive (and act upon) the information broadcast by $A_1$.
Specifically, if $y(t,j) = (y_1,\ldots,y_p)$, then to listen to $y_k$, $k=1,\ldots,p$, the state $x_2(t,j)$ of $A_2$ must satisfy some boolean `listening' condition $c(x_2,k)$.
For example, every agent $A_1$ must transmit its visual appearance - it can't become invisible.
For another agent $A_2$ to perceive it, $A_2$ must be close enough and with a line of sight to $A_1$. 
Then 
\[c(x_2) \equiv ||x_1 - x_2 || \leq d \land \forall i\neq 1,2, A_1,A_2,A_i \text{ not aligned}\]

This model of communication subsumes traditional models of processes that communicate via shared variables, like that in \todo[inline]{insert charon citation}, and generalizes it by imposing conditions under which communication is possible, rather than allow communication at all times.

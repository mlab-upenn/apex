\subsection{Decomposing a mission into scenarios and agents}
\label{scenarios and agents}
To motivate our thinking about the problem of safe autonomous navigation, we consider the case of a trip between two designated points.
On such a trip, the autonomous car will face a number of situations, or \emph{scenarios}, which it must know how to recognize, enter, negotiate, and exit in a safe and timely manner.
For example, the car will encounter roundabouts, 
traffic lights and four-way intersections, 
on-ramps to highways, 
lane changes, 
pedestrians,  
and various traffic signals like speed limits, school zones, etc.
We therefore think of a driving mission as an \emph{ordered sequence of scenarios}.
The exact sequence that will be encountered is not known ahead of time. 
%For example, it is not known whether some road has a large pothole that must be avoided.
Moreover, it is clear that some scenarios, like traffic lights, will be encountered more than once, albeit with minor site-specific differences. 
The key observation however is that the diversity of scenarios is, to a first order of approximation, finite. 
That is, there is a recognizable, finite set of scenario \emph{types} that is sufficient to describe most autonomous navigation missions.
The remaining variability among scenarios can be parametrized: e.g., the number of cars at a roundabout, or the current speed limit.

To formally define a scenario, we first outline its elements.
First, within each scenario, we can recognize a recurring set of entities: 
the autonomous vehicle whose operation we seek to verify (a.k.a. the \emph{ego} vehicle), the other vehicles on the road, pedestrians, traffic signage, and the road network itself. 
We will refer to these as \emph{agents}: an agent is an entity that functions continuously and autonomously in the environment, and whose presence can be sensed by other agents. 
\todo[inline]{main literature for agents}
Formally, we recognize the following set of four agent \emph{types}, whose semantics are given in the next section.
\[\agentTypeSet = \{\texttt{vehicle, pedestrian, road, trafficSignage}\}\]
Each agent type can have many (parametrized) instances, e.g. $\texttt{trafficSignage}$ can have instances speedLimit(70mph), speedLimit(25mph), HOVLane(3pm-6pm).
We denote the set of instances of agents in $\agentTypeSet$ by $I(\agentTypeSet)$.
Clearly, $I(\agentTypeSet)$ is infinite.
When we just speak of an agent, we mean an agent instance.

Formally, we define a scenario instance as follows. 
The elements of a scenario will be formally defined in the following sections.
\begin{defn}[Scenario instance]
	A scenario instance is a tuple $(\agentInstanceSet,\behavior, \lawSet, \Phi, Init, \exitConditions)$ where
\begin{itemize}
	\item $A$ is a collection of agent \emph{instances} from the set $I(\agentTypeSet)$.
	The set $\agentInstanceSet$ always includes the ego vehicle, i.e., the system whose behavior we want to verify.
	%
	\item $\behavior = \{\behavior_a \sut a \in \agentInstanceSet \setminus \{egoVehicle\}\}$ is a bounded-time behavior description for each of the other agent instances.
	Describing the behavior $\behavior_a$ of an agent instance requires us to formally describe an agent. 
	We do so in Section \ref{HCHA}.
	What assumptions we make on the behavior of other vehicles is captured in $\behavior$.
	%
	\item $\lawSet = \{l_0,\ldots,l_p\}$ is a finite set of $p \in \Ne$ traffic laws. 
	A law $l_i$ is a temporal logic formula indicating how the law constrains the behavior of the ego vehicle, but not the other agents. 
	I.e. it is a specification on the ego vehicle that must be satisfied. 
	% 
	\item $\Phi$ is a set of goals that must be met by the ego vehicle while in this scenario. 
	Now $\Phi$ and $\lawSet$ may both be expressed as (temporal) logic specifications on the system's behavior and therefore may be grouped together as one set.
	However it helps to keep these two aspects of the scenario separate: 
	that laws are constraints on the vehicle's behavior, and scenario goals are objectives to be met.
	%
	\item $Init$ is an initialization of the scenario, which defines a valid initial set of states for the agents when the scenario starts. $Init$ is formally defined in the next section.
	% 
	\item $\exitConditions$ is a set of condition describing how and when the scenario ends.
	The conditions are described as temporal logic formulae with atomic propositions on the states of the agents in the scenario.
	The formulae are required to be satisfiable by finite prefixes.
	That is, for any formula $\formula \in \exitConditions$, if there exists a trace $\sttraj$ of the system that satisfies $\formula$, then there exists a finite prefix of $\sttraj$ that satisfies $\formula$.
\end{itemize}
Let $\Sc = \{s_0,s_1,\ldots,s_{N-1}\}$ be a set of scenario instances. 
Then \textbf{a mission} $M$ is a finite string on $\Sc$, $M \in \Sc^*$.
\end{defn}

\todo[inline]{mission as an automaton?}

The logic in which a law $l_i$ or exit condition $\formula_i$ is expressed will depend on the formalism used to model the scenario's agents. 
We will have more to say about this in the following sections.

Because a mission is a sequence of scenarios, if we can verify the safety of the autonomous system's behavior in each scenario,
and compose the scenarios in a safe manner, 
then we have verified that the mission is safe.
In the rest of this paper we formalize the correct composition of scenarios and illustrate it with a case study in Section \ref{caseStudy}.
Note that in this paper, we do not study how to \emph{generate} missions for verification. 
Rather we assume that we are given mission to be verified. 
The problem of mission generation and verification coverage will be the subject of future research.

\begin{prob}[Correct composition of scenarios]
	
	\end{prob}


\begin{exmp}[Lane change continued]
	The mission can be decomposed into the following scenarios:
	$M = $ DriveStraight, ChangeLane, PassOnTheLeft, StopAtFourWayIntersection.
	Each of these scenarios contains two instances $v_{ego}$ and $v_2$ of the \texttt{vehicle} agent type, one \texttt{road} agent instance, no \texttt{pedestrian}s and one \texttt{trafficSignage} agent.
	The behavior $\behavior_{v_2}$ is given by a sequence of reach sets as detailed in Section \ref{otherAgents};
	briefly, $\behavior_{v_2}$ gives the area occupied by $v_2$ at any moment in time.
	An example applicable law, expressed in discrete time LTL, is $l \defeq \always(position_{ego} = a \implies X position_{ego} \geq a)$, to indicate that backing up is not allowed.
	Note that LTL is not ideally suited for this requirement since we need one formula per speed threshold $a$. A more concise logic (e.g., TPTL \cite{alur94_really}) would allow us to directly say $l \defeq \always(position_{ego}(t+1) \geq position_{ego}(t))$, while still being reasonably computationally tractable.
		
	There are two exit conditions for the DriveStraight scenario: either the ego vehicle reaches the end of the current road segment (i.e. , it reaches the intersection). 
	Or, it gets too close to the leading vehicle $v_2$, in which case it must exit and scenario ChangeLane begins.
	The $Init$ set for ChangeLane is the union of two sets $Init_1$ and $Init_2$.
	$Init_1$ is the set of 2D positions on the road where $v_ego$ is behind $v_2$ and within some distance $d_{min}$ of $v_2$.
	$Init_2$ is the set of 2D positions of $v_{ego}$ that are some distance $d_{turn}$-close to the intersection.
	I.e., we consider the lane change to be provoked by excessive proximity to the leading vehicle $v_2$, or by proximity to the intersection.	
\end{exmp}

It should be noted that the initialization of a scenario may have a mission-dependent part, like $Init_2$ in the example.
This highlights the need to do verification \emph{in the context of the mission}, as the context provides information on what needs to be verified.
	
%	E.g an instance of scenario $s_0$ (Roundabout) might have the agents 
%	\begin{eqnarray*}
%		\agentInstanceSet = \{&egoVehicle, otherVehicle1, rndAbt(3), \\ 
%		& speedLimit(25mph)\} \subset I(\agentTypeSet)
%	\end{eqnarray*}
%	where $egoVehicle$ and $otherVehicle1$ are instances of type $\texttt{vehicle}$, 
%	$rndAbout(n)$ is an instance of $\texttt{road}$ indicating a round-about with $n$ entry points (which are also exit points),
%	and $speedLimit(Vmph)$ is an intance of $\texttt{trafficSignage}$ indicating a speed limit sign that reads $V$ mph.
%	Another scenario instance has the agents
%	\begin{eqnarray*}
%		\agentInstanceSet = \{&EgoVehicle, OtherVehicle1, rndAbt(2), \\ 
%		& speedLimit(35mph)\} \subset I(\agentTypeSet)
%	\end{eqnarray*}
%	
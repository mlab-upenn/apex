\section{Design entry and the tool bus}
\label{designEntryAndToolBus}

\subsection{The tool bus}
\label{tool bus}
One of our goals in APEX is to formally verify the safety of the autonomous system's operation in all scenario types.
As we saw in Section \ref{safety}, the imperative to be safe is expressed by different logic properties in the different scenarios types.
To verify these properties, we must first decide on the appropriate formalism(s) in whic to express them (e.g. LTL, CTL, or MTL?) 
and the corresponding formalism for describing the autonomous system under verificaiton.
Secondly, we must decide on which tool to run to verify the property.
In general, we may expect that different properties, and different levels of abstraction at which they are verified, will require (or be better verified) using different formalisms and tools. 
For example, verifying a property of the continuous-time dynamics may be done using {\staliro},
whereas verifying a property of the discrete mission planner may be best done using UPPAAL.
This motivates the creation of a translator from our rich HCHA representation of the scenario instance to the different formalisms that are deemed useful. 
For example, finite automata, timed automata, weighted timed automata,...etc.
See Fig.~\ref{fig:toolbus}.

\begin{figure}[tb]
	\centering
		\includegraphics{figures/fly.eps}
	\label{fig:toolbus}
	\caption{The APEX tool bus.}
\end{figure}

\todo[inline]{put figure of tool bus}

The representation of a scenario and its agents as HCHA is rich enough to allow the translation in a traceable manner.

Because the safety imperative is expressed by a set of logic formulae in different scenarios
The HCHA formalism is rich, but not every property to be verified requires this level of detail.
The key thing is that it is rich enough to be translated into simpler formalisms that have FV tools and can support the verification of properties of interest.
The translation must be formally valid: if HCHA -> Timed automata, say, and TA |= property, then it must hold that HCHA $\models$ property.

Therefore, tool bus.



\subsection{Scenario authoring tool and Domain-Specific Language}
\label{dsl}
How are the scenarios created? DSL and scenario authoring tool.
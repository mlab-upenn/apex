\subsubsection{Properties of the abstraction chain}
\label{thmAbstractionLevels}
\begin{thm}
	Let $(\Sys,\Vc,\Pc,\{\Sys_i\})$ be a HCHA.
	Let $\hstraj =(\mode,\sttraj) \in \Sc_{\Sys}$ be a solution of CHA $\Sys$ with continuous part $\sttraj$.
	Then for every $1 \leq i \leq a$, 
	there exists a solution 
	$\hstraj^{(i)} = (q,\sttraj^{(i)}) \in \Sc_{\Sys_i}$ 
	such that $\sttraj^{(i)} = \proj{\sttraj}{V_i}$.
	That is, for every $(t,j) \in \dom \hstraj$, there exists $(t,j') \in \dom \hstraj^{(i)}$
	s.t. $\mode(t,j) \in q(t,j')$
	and	$\sttraj^{(i)}(t,j) = \proj{\sttraj(t,j')}{V_i}$.
\end{thm}

\begin{proof}
	We prove this for $i=1$. The general case follows by induction.
	Fix $i=1$, and write $V$ for $V_i$ in this proof.
	Let $\sttraj$ be as stated, and define 
	$y: \dom \sttraj \rightarrow \reals^{n_i}$ by 
	$y(t,j) = \proj{\sttraj(t,j)}{V}$ for all $(t,j) \in \dom \sttraj$.
	We will construct from $y$ a hybrid arc that satisfies the conditions of a solution of $\Sys_i$.
	
	Let $I_j$ be an interval in $\dom \hstraj$ with non-empty interior, and take $t \in I_j$ s.t. 
	$\dot{\sttraj}(t,j) \in F_\mode(\sttraj(t,j))$. 
	Then $\dot{y}(t,j) = d(\proj{\sttraj(t,j)}{V})/dt = \proj{\dot{\sttraj}(t,j)}{V} \in \proj{F_\mode(\sttraj(t,j))}{V} \subset F_q(y^{\uparrow}(t,j))$.	
	For all $t \in \text{int}I_j$, $y(t,j) = \proj{\sttraj(t,j)}{V} \in \proj{Inv(\mode)}{V} = Inv(q)$.
	Moreover, for all $t$ different from the endpoints of $I_j$, $q(t,j) = [\mode]$ and so $\dot{q}(t,j) =0$.
	Thus condition 3 is satisfied for $y$.
	
	Now consider the jumps: let $t_r = \sup I_J$. 
	If the jump at $(t_r,j)$ leads from $\mode$ to a location $\mode'$ such that $[\mode']$ is not $q$, then it can be shown that this is also a jump for $y$.
	However, if the transition is \emph{internal} to mode $q$, i.e. 
	\[[\mode]=[\mode']\]
	then this violates condition 4 since $y(t_r,j)$ is not in a guard set of $\Sys_i$.
	Moreover, because of the reset that occurs at $(t_r,j)$, if $\proj{\sttraj(t_r,j+1)}{V}\neq \proj{\sttraj(t_r,j)}{V}$, then $y$ is not absolutely continuous as a function of time within each mode.
	To get around these difficulties, we invoke the last property defining an abstraction chain.
	Namely, that the surviving variables at each level are not affected by resets on internal transitions.
	This removes the second difficulty above since now $y(t_r,j+1) = y(t_r,j)$.
	
	We define $\sttraj^{(i)}$ by erasing the internal transitions from the domain of $y$.
	Let $\Jc$ be the set of indices of internal transitions, that is, $\Jc = \{j \sut \exists t.(t,j) \in \dom \sttraj, (t,j+1) \in \dom \sttraj, [\mode(t,j)]=[\mode(t,j+1)]\}$.		
	
	\todo[inline]{TBC}
\end{proof}

\todo[inline]{initial points}

\begin{thm}
	Let $(\Sys,\Vc,\Pc,\{\Sys_i\})$ be a HCHA.
	Let $y \in \Sc_{\Sys_i}$ be a solution of CHA $\Sys_i$ for some $i$.
	Then either there exists $\sttraj \in \Sc_\Sys$ s.t. $y = \proj{\sttraj}{V_i}$,
	or $y$ can be refined to such a trajectory in at most $|\modeSet|\cdot n$ steps,
	where $\modeSet$ is the location set of $\Sys$ and $n$ is its dimension.
\end{thm}

\begin{proof} (Sketch)
	Assume that a solution $\sttraj \in \Sc_\Sys$ as stated does not exist.
	Broadly speaking, this can happen because $\Sys_i$ can choose a sequence of continuous dynamics that can not be selected by $\Sys$, since the dynamics of $\Sys_i$ are larger (in the sense of Def.~\ref{def:HCHA}) than those of $\Sys$.
	More specifically, there are two ways this can happen:
	\begin{enumerate}
		\item Either, within the same $I_j$ in the domain of $y$, $\dot{y}$ obeys two different dynamics $F_\mode$ and $F_{\mode'}$ for $\mode,\mode' \in q \in \modeSet_i$.
		In general, this can not be produced by a $\Sys$ trajectory.\footnote{Unless it so happens that some element of $y(t_j,j)^{\uparrow}$ is in $\guard(\mode,\mode')$ (where $t_j$ is the transition time). }
		\label{item:modechange}
		%
		\item Or, within the same $I_j$ in the domain of $y$, $\dot{y}$ obeys two different dynamics: 
		over $[t,t') \subset I_j$ it obeys $F_\mode(\sttraj(t))$ where $\sttraj$ is a $\Sys$ solution that extends $y$, i.e. $\sttraj = [z,y] \in \reals^n$.
		And at $t'$ it obeys $F_\mode(x')$ where $x' \in y^\uparrow(t') \setminus \{\sttraj(t')\}$. 
		\label{item:xchange}		
	\end{enumerate}
	
	Note that if we exclude the above two cases, then the discrete dynamics (i.e. the transitions) of $\Sys_i$ can always be reproduced by $\Sys$.
	
	Now suppose Case \ref{item:modechange} happens: then we refine the partition $\Pc$ by creating a separate part for $\mode'$. I.e. if $\mode' \in P_1$, then we refine $\Pc$ to 
	$\Pc' = \Pc \setminus \{P_1\} \cup (P_1 \setminus \mode') \cup \{\mode'\}$.
	This guarantees that there won't be invisible dynamics changes caused by the (previously internal) transition $(\mode,\mode')$.
	Since there are $|\modeSet|$ locations, this refinement is applied at the most $|\modeSet|$ times.
	Doing the refinement $|\modeSet|$ times corresponds to preserving the discrete dynamics perfectly.
	
	Now suppose Case \ref{item:xchange} happens: then we must eliminate the invisible dynamics changes that occur because of the projection of the state vector onto $V_i$.
	Let $W$ be the set of indices where $\sttraj(t')$ and $x'$ differ. 
	First, because $\sttraj$ extends $y$, and therefore agrees with it over $V_i$, $W$ and $V_i$ don't intersect. 
	Thus $W = \{i_1,i_2,\ldots,i_m\}$ consists entirely of `dead' variables (i.e. non-surviving).
	We refine $\Sys_i$ as follows: promote the smallest index in $W$ to surviving at level $i$: i.e. define $V_i' = V_i \cup \{i_1\}$.
	Then the dynamics of $\Sys_i$ must be refined accordingly, namely 
	$F'_q(y) \defeq \proj{\cup_{\mode \in q,x \in y^\uparrow}F_\mode(x) }{V_i'}$.
	Thus the solutions of $\Sys_i$ can no longer choose the dynamics $F_\mode (x')$ which led to Case \ref{item:xchange} above.
	Of course, the rest of the variables in $W$ could still cause this problem.
	But we refine one index at a time, since every refinement produces a new set of dynamics, which might lead to a different violation, and make others impossible. 
	I.e., the set $W$ depends on the current $V_i$ and is not static. 
	Since there are $n$ indices, and a new index is added to $V_i$ with each refinement, this can happen a maximum of $n$ times. Surviving all indices (i.e., $V_i' = \{1,\ldots,n\}$) corresponds to preserving the continuous dynamics perfectly.
	
	Thus there is a total of $|L|n$ refinements possible before every $\Sys_i$ trajectory is the projection of a $\Sys$ trajectory.	
\end{proof}

	Note that this maximum number of refinements is \emph{independent of the level of abstraction $i$}.
\begin{thm}
	Let $(\Sys,\Vc,\Pc,\{\Sys_i\})$ be a HCHA.
	Let $\hstraj =(\mode,\sttraj) \in \Sc_{\Sys}$ be a solution of CHA $\Sys$ with continuous part $\sttraj$.
	Then for every $1 \leq i \leq a$, 
	there exists a solution 
	$\hstraj^{(i)} = (q,\sttraj^{(i)}) \in \Sc_{\Sys_i}$ 
	such that $\sttraj^{(i)} = \proj{\sttraj}{V_i}$.
	That is, for every $(t,j) \in \dom \hstraj$, there exists $(t,j') \in \dom \hstraj^{(i)}$
	s.t. $\mode(t,j) \in q(t,j')$
	and	$\sttraj^{(i)}(t,j) = \proj{\sttraj(t,j')}{V_i}$.
\end{thm}

\begin{proof}
	We prove this for $i=1$. The general case follows by induction.
	Fix $i=1$, and write $V$ for $V_i$ in this proof.
	Let $\sttraj$ be as stated, and define 
	$y: \dom \sttraj \rightarrow \reals^{n_i}$ by 
	$y(t,j) = \proj{\sttraj(t,j)}{V}$ for all $(t,j) \in \dom \sttraj$.
	We will construct from $y$ a hybrid arc that satisfies the conditions of a solution of $\Sys_i$.
	
	Let $I_j$ be an interval in $\dom \hstraj$ with non-empty interior, and take $t \in I_j$ s.t. 
	$\dot{\sttraj}(t,j) \in F_\mode(\sttraj(t,j))$. 
	Then $\dot{y}(t,j) = d(\proj{\sttraj(t,j)}{V})/dt = \proj{\dot{\sttraj}(t,j)}{V} \in \proj{F_\mode(\sttraj(t,j))}{V} \subset F_q(y^{\uparrow}(t,j))$.	
	For all $t \in \text{int}I_j$, $y(t,j) = \proj{\sttraj(t,j)}{V} \in \proj{Inv(\mode)}{V} = Inv(q)$.
	Moreover, for all $t$ different from the endpoints of $I_j$, $q(t,j) = [\mode]$ and so $\dot{q}(t,j) =0$.
	Thus condition 3 is satisfied for $y$.
	
	Now consider the jumps: let $t_r = \sup I_J$. 
	If the jump at $(t_r,j)$ leads from $\mode$ to a location $\mode'$ such that $[\mode']$ is not $q$, then it can be shown that this is also a jump for $y$.
	However, if the transition is \emph{internal} to mode $q$, i.e. 
	\[[\mode]=[\mode']\]
	then this violates condition 4 since $y(t_r,j)$ is not in a guard set of $\Sys_i$.
	Moreover, because of the reset that occurs at $(t_r,j)$, if $\proj{\sttraj(t_r,j+1)}{V}\neq \proj{\sttraj(t_r,j)}{V}$, then $y$ is not absolutely continuous as a function of time within each mode.
	To get around these difficulties, we invoke the last property defining an abstraction chain.
	Namely, that the surviving variables at each level are not affected by resets on internal transitions.
	This removes the second difficulty above since now $y(t_r,j+1) = y(t_r,j)$.
	
	We define $\sttraj^{(i)}$ by erasing the internal transitions from the domain of $y$.
	Let $\Jc$ be the set of indices of internal transitions, that is, $\Jc = \{j \sut \exists t.(t,j) \in \dom \sttraj, (t,j+1) \in \dom \sttraj, [\mode(t,j)]=[\mode(t,j+1)]\}$.		
	
	\todo[inline]{TBC}
\end{proof}

\todo[inline]{initial points}

\begin{thm}
	Let $(\Sys,\Vc,\Pc,\{\Sys_i\})$ be a HCHA.
	Let $y \in \Sc_{\Sys_i}$ be a solution of CHA $\Sys_i$ for some $i$.
	Then either there exists $\sttraj \in \Sc_\Sys$ s.t. $y} = \proj{\sttraj}{V_i}$,
	or $y$ can be refined to such a trajectory in at most $|\modeSet|\cdot n$ steps,
	where $\modeSet$ is the location set of $\Sys$ and $n$ is its dimension.
\end{thm}

\begin{proof}
	Write $y$ for Assume that a solution $\sttraj$ as stated does not exist.
	There are two reasons why this might happen:
	\begin{itemize}
		\item Either, iwithin the same $I_j$ in the domain of $y$, $\dot{y}$ obeys two different dynamics $F_\mode$ and $F_{\mode'}$ for $\mode,\mode' \in q \in \modeSet_i$.
		In general, this can not be produced by a $\Sys$ trajectory.\footnote{}
	\end{itemize}
\end{proof}
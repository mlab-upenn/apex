\begin{abstract}
Autonomous and semi-autonomous systems are now an accepted part of the economy and society, whether they are autonomous robots in shipping warehouses or autonomous appliance robots in the home.
The success of these robots can be attributed, in part, to the fact that either they function in highly controlled environments where their interactions with other agents are structured, or their tasks are simple enough to accommodate unstructured and unpredictable environments.
In this paper, we develop a framework for formally verifying the safety of autonomous systems operating in unstructured environments in the presence of other unpredictable agents. 
Using autonomous vehicles as a running example, 
we present a decomposition of an autonomous system's mission into scenarios and agents, and a rich representation of these two elements in terms of hierarchical communicating hybrid automata. 
Our goal is to formally verify that the behavior of the autonomous system is safe in a given scenario, which possibly involves other unpredictable (nondeterministic) agents.
Therefore, we include in our framework a translator from this rich representation into the formalisms of various formal verification tools, to enable the use of the most appropriate tool for a given verification task. 
The fact that multiple tools are needed is illustrated with the verification of the safety of a lane change maneuver, which requires the use of a timed automaton verification tool, and a rechability tool.
Another goal is to allow control engineers to use these formal tools as part of their design process. 
Therefore, we develop a scenario authoring tool, and a domain-specific language with formal semantics, for enabling rapid prototyping and verification of controllers' designs.
\end{abstract}
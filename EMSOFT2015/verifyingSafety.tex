\section{Verifying the safety of abstract plans for discrete maneuvers}
\label{verifyingSafety}

{\it To verify the lane change scenario given in the previous sections, we must integrate two formalisms: timed automata for correctness of the controller's actions (is the scenario's goal achieved?), and reachability for safety of the controller's actions.} 
We have already defined the basic representation of an autonomous driving scenario as a HCHA. 
A single mode of the HCHA can be viewed from two perspectives: discrete as represented by a state in a timed automaton, and continuous as represented by a plant and systems of nonlinear ODEs. 
Interaction between these representations occurs when guard conditions in the discrete representation are encountered. 
One way of reasoning about the goal specifying behavior of the scenario controller is to investigate the safety and performance of the timed automaton using model checking methods.

{\it Thus, we translate the scenarios's HCHA to timed automata.} 
Given the HCHA it is possible to create a map which has abstracts the system to a timed automaton on which we can perform verification tasks. 
\begin{thm}
	Let \(X\) and \(Y\) be vector fields on \(M\) and \(N\) respectively and let \(\phi: M \rightarrow N\) be a smooth surjective map. Then vector field \(Y\) is an abstraction of vector field \(X\) with respect to \(\phi\) iff for every integral curve \(c\) of \(X\), \(\phi \circ c\) is an integral curve of \(Y\). \cite{Pappas1998} 
\end{thm}
If we consider that the abstraction of the system is overapproximate, then we may conclude that it is sufficient to check safety properties on the abstraction \cite{Pappas1998}. A result that returns safe must be safe. However, given a counterexample to a proposition specifying a safety property one cannot be sure that it is not spurious. Thus, it would also be natural to attempt to refine abstraction to remove any spurious counter examples. We will address this notion of counterexample guided abstraction refinement shortly \cite{Clarke2003}, but first it is essential to discuss the hierarchical aspect of the HCHA.

Hierarchical controls are commonly used to reason about systems which control realistic levels of complexity. Thus, the scenario controller as implemented in the real system will interact with the motion planner. When considering the motion planning controller, reachability is the formalism for which it is convenient to explore the feasibility and safety of the planned trajectories. Reachability analysis determines safety by checking the intersection of reachable sets of the underlying dynamics with unsafe regions. This investigation leads to the fundamental question:{\it if the discrete controller is safe, and the motion planner is safe, is the hierarchical composition of the two safe?} Fortunately, the answer to this question can be yes. 

\begin{prop}
	Let \(X_1\), \(X_2\), \(X_3\) be vector fields on manifolds \(M_1\), \(M_2\), and \(M_3\) respectively. If \(X_2\) is an abstraction of \(X_1\) with respect to the map \(\phi_1 : M_1 \rightarrow M_2\) and \(X_3\) is an abstraction of \(X_2\) with respect to map \(\phi_2 : M_2 \rightarrow M_3\) then \(X_3\) is an abstraction of \(X_1\) with respect to abstracting map \(\phi_1 \circ \phi_2\) \cite{Pappas1998}
\end{prop}

{\it Thus, once each subsystem can be verified, we must map back the verification results to the HCHA formalism.}

Going a step beyond the results in \cite{Pappas1998} We propose that the relationship between verification, in the sense of logical analysis of a system, and control, in the sense of the evolution of state variables described by nonlinear ODEs, is so tightly intertwined that new approaches to system design are necessary. By viewing the activities of correct by construction controller design and verification as tightly coupled processes instead of distinct options we can design solutions for richer, more complex problems. Returning to the lane change case study, we summarize state of the art work in the area of provably safe nonlinear hybrid systems verification and control.

One approach to designing safe lane change and evasive vehicle maneuvers explores the use of reachability to generate an automaton which switches between motion primitives in an offline phase. Online the precomputed reachable sets of motion primitives are compared to the reachable sets of the other agents. It is possible to perform the reachable set calculations online because there is less non-determinism about the initial state of the other agents, there is no tracking controller in the loop, and the dynamics are simpler.\cite{Hess2014}

Another approach explores the use of reachability entirely online as a means of verifying and controlling vehicle maneuvers. In order to ease restrictions on computation timing linearized vehicle dynamics are used when considering the target vehicle. Perturbations representing external conditions such as road surface quality and sensor noise are added in order to improve the accuracy of the linearized model. Current run times are on the order of 4-5 seconds making this approach impractical for unexpected events. Furthermore the linearization requires the overapproximation of the system making the results conservative.\cite{Althoff2014b}

Notions of reachability have also been used within the CEGAR framework introduced by Clarke et al. The CEGAR framework for automatic refinement of system abstractions checks to see whether counterexamples generated during model checking of an abstraction of the system of interest are real by examining  whether the counterexample is spurious on a less abstract version of the system. Through this process the abstract version of the system can be iteratively refined.\cite{Clarke2003}

The CEGAR framework can be extended as an alternative to the online reachability analysis presented by Althoff and Hess. Through the combination of correct by construction controller synthesis with approximate solutions to constrained reachability problems feasible trajectories are computed for nonlinear systems. While the authors present a solution to the motion-planning problem for nonlinear systems, they do not consider other dynamic agents. Furthermore, the authors' decision procedure is not complete. Finally, the presented method of refining the control automaton relies on enumerating and ranking solutions under a cost function which is not well defined in their work.\cite{Wolff2013}

In our problem we consider several other vehicles in the environment with simple 2-D kinematics and the target vehicle with the nonlinear bicycle model dynamics. While in general the reachability problem for non-linear hybrid systems is undecidable. Recent work has shown that if a user is able to specify a bound on the ODE which is an arbitrary positive rational number \(\delta\), then propositions in first order logic over a region \textit{unsafe} can be evaluated \cite{gao2014delta}. Specifically a system, \(H\), is \(safe\) if it cannot reach \(unsafe\), and \(H\) is \(\delta-unsafe\) if \(H^d\) can reach \(unsafe\). 

The unsafe region of a driving scenario as the intersection between the reachable set of the target vehicle and the reachable set of the other vehicles in the scenario over a specified time horizon \(\Delta t\). We note that using the concept of \(\delta-reachability\) does not have any affect on the quality of our results when compared with traditional approaches to reachability. If the system is safe then a logical proof is generated showing that the original system \(H\) cannot reach the unsafe region. Alternately if the \(\delta-reachability\) analysis returns unsafe then the answer means that an overapproximated version of the system, \(H^{\delta}\) reaches the unsafe region. Clearly, one can adjust \(\delta\) so as to discover the robustness of the system. 

\todo [inline] {Houssam: add information about sampling-based motion planning techniques.} 
\todo [inline]{Matt: add in something about Sarah Loos' work on thm proving}

\todo[inline]{literature review}

\todo[inline]{Pending the lit review, might be worth investigating two approaches we discussed:
if the boxes of the partition used by controller are over-approximations of the reach set alphabet, and model checker returns SAT, then must compute reachability for every satisfying trace. Obviously, can get expensive.

Else, we can partition the state space using the reach set alphabet as explained on the board, so the model checker's answer needn't be concretized: by construction it gives a sound and complete answer.

I'd be very surprised if these two don't have antecedents in the literature, they're natural enough...

Because ours is a verification task, it's not up to us to dictate the controller's view of the world.
}

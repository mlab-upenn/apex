\section{Case study}
\label{caseStudy}
\subsection{Scenario Description}
Lane change manuever scenario occurs on a 2 lane road with the target vehicle and 1 environmental vehicle. The target vehicle must change lanes in order to get into the left turn lane before the end of the scenario.
\begin{figure}[tb]
	\label{fig:discreteview}
		\includegraphics[scale=.4]{scenario.png}
	\caption{Pictorial description of target vehicle scenario}
\end{figure}
\subsection{Target Vehicle Model}
\begin{enumerate}
	\item Target vehicle has non linear dynamics described by bicycle model.
	\item Trajectories are defined for right to left lane switch and left to right lane switch.
	\item Hybrid system modes are described by drive forward and lane change.
	\begin{enumerate}
		\item Both lane change modes have the same dynamics and controller but implement a different motion primitive (trajectory).
	\end{enumerate}
	\item Target vehicle has non linear dynamics described by bicycle model:
	\begin{equation}
	\begin{aligned}
	\ddot{\beta}=\left(\frac{C_rl_r-C_fl_f}{mv^2} \right)\dot{\psi}+\\
	\left(\frac{C_f}{mv} \right)\delta-\left(\frac{C_f+C_r}{mv} \right)\beta+y_{\beta}
	\end{aligned}
	\end{equation}
	\begin{equation}
	\begin{aligned}
	\ddot{\psi}=\left(\frac{C_rl_r-C_fl_f}{I_z} \right)\beta-\\
	\left(\frac{C_fl_f^2-C_rl_r^2}{I_z} \right)\left(\frac{\dot{\psi}}{v} \right)+
	\left(\frac{C_fl_f}{I_z} \right)\delta+y_{\dot{\psi}}
	\end{aligned}
	\end{equation}
	\begin{equation}
	\dot{v}=a_x+y_v
	\end{equation}
	\begin{equation}
	\dot{s_x}=v\cos{\beta+\psi}+y_{s_x}
	\end{equation}
	\begin{equation}
	\dot{s_y}=v\sin{\beta+\psi}+y_{s_y}
	\end{equation}
	\begin{equation}
	\dot{\delta}=v_w+y_d
	\end{equation}
	
	\(C_f,C_r\) and \(l_f, l_r\) describe respectively the cornering stiffness and distances from the center of gravity to the axles. \(I_z\) is the moment of inertia and \(m\) is the vehicle mass. \(\beta\) is the slip angle at the center of mass, \(\psi\) is the heading angle, \(\dot{\psi}\) is the yaw rate, \(v\) is the velocity, \(s_x\) and \(s_y\) are the x and y positions, and \(\delta\) is the angle of the front wheel. In the formulation of [6], the inputs to the system are \(a_x\), the logitudinal acceleration, and \(v_w\) the rotational speed of the steering angle. The \(y\) terms represent disturbances to the system. For example \(y_{\beta}\) and \(y_{\dot{\psi}}\) represent disturbances to the slip angle at the center of mass and the yaw rate. 
	\item Trajectories are defined for right to left lane switch and left to right lane switch.
	\item Hybrid system modes are described by drive forward and lane change.
	\begin{enumerate}
		\item Lane change is hierarchical and has sub-modes, right to left, straight, and left to right.
		\item Each submode has same dynamics but implements a different motion primitive.
	\end{enumerate}
	\item Final control equations:
	\begin{equation}
	\begin{aligned}
	v_w=k_1(cos{(\Psi_d)}(s_{y,d}-s_y-w_y)\\
	-sin{(\Psi_d)}(s_{x,d}-s_x-w_x))\\ 
	+k_2(\Psi_d-\Psi-w_{\Psi}) \\
	+k_3(\dot{\Psi_d} -\dot{\Psi}-w_{\psi})\\
	-k_4(\delta-w_{\delta})
	\end{aligned}
	\end{equation}
	\begin{equation}
	\begin{aligned}
	a_x=k_5(cos{(\Psi_d)}(s_{x,d}-s_x-w_x)\\
	+sin{(\Psi_d)}(s_{y,d}-s_y-w_y))\\
	+k_6(v_d-v-w_v)
	\end{aligned}
	\end{equation}
\end{enumerate}
\begin{figure}[tb]
	\label{fig:discreteview}
	\centering
	\includegraphics[scale=.5]{hybrid.png}
	\caption{Hybrid automata describing lane change}
\end{figure}
\subsection{Environmental Vehicle Model}
Environmental vehicle has simple 2D dynamics, no specific controller, non-determinism describes state evolution. When considering other agents (especially vehicles) in a scenario such non-linear dynamics are completely unnecessary. The primary reasoning is that the uncertainty about the state of other vehicles is not do to poor model parameter identification, but rather unknown inputs [2]. The dynamics of the other agents are described by equations (7) and (8) as in [3]:
\begin{equation}
\dot{s_x}=v\cos{\beta}
\end{equation}
\begin{equation}
\dot{s_y}=v\sin{\beta}
\end{equation}

\begin{enumerate}
	\item Environmental vehicle must drive forwards.
	\item Environmental vehicle must not change lanes if other lane is occupied.
	\item At every step the environmental vehicle can apply +/- some specified level of acceleration.
	\item At every step environmental vehicle can switch lanes if condition 2 is not violated.
\end{enumerate}
Environmental vehicle has simple 2D dynamics, no specific controller, non-determinism describes state evolution.
\begin{enumerate}
	\item Environmental vehicle must drive forwards.
	\item Environmental vehicle must not change lanes if other lane is occupied.
	\item At every step the environmental vehicle can apply +/- some specified level of acceleration.
	\item At every step environmental vehicle can switch lanes if condition 2 is not violated.
\end{enumerate}
\subsection{dReal Model}
\begin{enumerate}
	\item 1 or 2 modes depending on lane change or passing.
	@@ -28,5 +27,4 @@ Passing manuever is scenario on 2 lane road, target vehicle, 1 environmental veh
	\item Plant controller is tracking controller which follows trajectory.
\end{enumerate}
\subsection{Controller with Partial Dynamics}
A finite transition system describing the passing manuever is given in UPPAAL (or NuSMV).
\begin{figure}[tb]
	\label{fig:discreteview}
	\centering
	\includegraphics[scale=.7]{lane_change_timed.png}
	\caption{Automata with partial dynamics describing scenario}
\end{figure}
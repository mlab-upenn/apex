\subsection{Communicating hierarchical hybrid \\automata}
\label{HCHA}
{\it As noted in the introduction, the execution of an autonomous plan involves many levels of abstraction. The verification must cover all these levels.}

{\it We use the formalism of hierarchical communicating hybrid automata (HCHA) to model agents in the scenario. }
\todo[inline]{Charon: make sure that the HCHA is implemented in Charon (or something close)}
\todo[inline]{Ptolemy}

\begin{defn}[Hybrid automaton]
	
	\end{defn}
	
The differential equations in each mode of the HA model the continuous-time closed-loop behavior of the low-level controlled systems, i.e., the mechanical systems of the car such as the powertrain. 
The modes themselves model the application of different control laws. 
Transitions between modes, modeled to be instantaneous, are caused by the satisfaction of certain guard conditions that depend on time, the current state, and the initial state of the system.

\begin{defn}[Hierarchical hybrid automaton]
	
\end{defn}
A hierarchical HA (HHA) is an HA, each of whose modes is itself an HA.
I.e., the dynamics in a given mode of the HHA are provided by an HA.
This hierarchy can be deepened to more than two levels, that is, the mode-embedded HA can itself be an HHA.


\begin{defn}[Communicating hybrid automata]
	
\end{defn}
A set of communicating HA (CHA) is a set of HA that \emph{occasionally} share state variables.
That is, each HA in a CHA has a `message board' to which it posts events about its behavior.
Other HAs in the CHA can subscribe and unsubscribe to it: 
subscribed HAs can make use of that information, whereas the rest can't. 
Subscription is not voluntary: it is triggered by the common evolution of the CHA.
For example, a traffic light is broadcasting its current state (Green, Red or Yellow).
A car won't subscribe to this message board until it is within line of sight of, and sufficient proximity to, the light.


\begin{defn}[Communicating hierarchical hybrid automata]
	\label{def:CHHA}
	
	
\end{defn}
Communicating hierarchical hybrid automata (CHHA) are simply a CHA where each component automaton is an HHA.



%Because autonomous systems are physical systems controlled by software, hybrid automata (HA) are a suitable formalism for modeling them.
%Because the complexity of autonomous systems is significant, they are typically designed at multiple levels of abstraction, 
%with different teams handling the design at different levels.
%E.g. the mapping team might design the algorithm for finding the waypoints along a shortest route from start to end.
%For this design, knowledge about, say, the powertrain control of the car is immaterial and abstracted away.
%Indeed, details about the road itself are abstracted away as well.
%Instead, it is represented as a directed graph.
%
%The resulting system's description is given at multiple levels of abstraction (or multiple levels of detail). 
%To model this situation, we adopt \emph{hierarchical} HA: each mode of the HA is itself an HA, down to some level.
%
%Finally because these HHA, each representing an agent, are sensed by other agents in the scenario, we must model this sensing. 
%We do so via the inputs to each HHA: given agent instance $a$, every other agent is associated with an input port on $a$.
%When that other agent is within sensing distance of $a$, that input port is occupied by that agent's sensed information.

{\it The behaviors of the non-ego agents are described, at the lowest level, via a possibly non-deterministic sequence of reach sets.}

{\it HCHA can be translated to various other formalisms. We consider the case of timed automata, and of ODEs (i.e. dynamical systems).}

\begin{exmp}[Lane change continued]
	We model the lane change scenario using HCHA as follows
	\end{exmp}

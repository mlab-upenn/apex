\subsection{What is safety?}
\label{safety}

{\it For an autonomous vehicle, safety is equivalent to not colliding with other agents (cars, pedestrians, and traffic signage), or entering unsafe regions of the environment, like opposing traffic lanes or obstacles.}

{\it In a given scenario, the safety imperative gives rise to more specific requirements. 
	We express these in a temporal logic: it allows formal verification, and applies to output traces regardless of the level of abstraction of the generating system. We can also express (non-safety) mission goals in temporal logic.}
%Intuitively, by specifying certain ways in which safety can be violated, we guide the model checker towards finding those ways, in effect constraining the search space.
%\todo[inline]{cite XOR (formalize => prove)? (in that order)}

{\it The specific logic we use depends on the property being expressed.}

\begin{exmp}[Lane change continued]
	The safety imperative in the lane change scenario can be expressed as the conjunction of the following LTL formulae on the grid world evolution and on the reach sets.
	
	The mission goal can be expressed as follows
\end{exmp}

\todo[inline]{Future: Can we take the absolute safety mandates and automatically specialize them to scenarios? I.e. given the set of state trajectories of the car, automatically derive, from the scenario, some constraints that constrain the search for the subset of trajectories that end in harming someone in that particular scenario?}
\todo[inline]{safety algebara?}